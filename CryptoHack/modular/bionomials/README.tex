\documentclass{article}
\usepackage[utf8]{inputenc}
\usepackage{amsmath}

\title{CryptoHack}
\author{Appaji Chintimi}
\date{October 2021}

\begin{document}

\maketitle

\section{Introduction}

First things first, By Symmetry, We can interchange the $c_1$ and $(2^p + 3^q)$.

In the expansion of $(x + y)^n$, We can ignore all terms except first and last. Because all the terms except first and last contain the product of $x$ and $y$, and Since $N = p * q$, their contribution to modulo $N$ is $0$. Combining all properties, we get following equations.

\begin{equation}
    ((2p)^{e_1} + (3q)^{e_1})  \equiv c_1 mod N
\end{equation}
\begin{equation}
    ((5p)^{e_2} + (7q)^{e_2})  \equiv c_2 mod N
\end{equation}

By compatibility with exponentiation, And using above property again (Ignoring middle terms), we get,
\begin{equation}
    ((2p)^{e_1e_2} + (3q)^{e_1e_2})  \equiv c_1^{e_2} mod N
\end{equation}
\begin{equation}
    ((5p)^{e_1e_2} + (7q)^{e_1e_2})  \equiv c_2^{e_1} mod N
\end{equation}

Let's define few variables for convenience. $a = 2^{e_1e_2}$, $b = 3^{e_1e_2}$, $x = c1^{e_2}$, $c = 5^{e_1e_2}$, $d = 7^{e_1e_2}$, $y = c_2^{e_1}$. Note that,

\begin{equation*}
    x = (ap^{e_1e_2} + bq^{e_1e_2}) mod N
\end{equation*}
\begin{equation*}
    y = (cp^{e_1e_2} + dq^{e_1e_2}) mod N
\end{equation*}

\begin{equation*}
    p = gcd(dx - by, N)
\end{equation*}
\begin{equation*}
    q = gcd(ay - cx, M)
\end{equation*}


\end{document}
